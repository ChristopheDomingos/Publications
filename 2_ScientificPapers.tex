\section*{Peer-reviewed scientific articles}

\begin{enumerate}

  \item[13.] Correia, J. P., Domingos, C., Witvrouw, E., Luís, P., Rosa, A., Vaz, J. R., & Freitas, S. R. (2023). Brain and muscle activity during fatiguing maximum-speed knee movement. \emph{Journal of Applied Physiology}. \url{https://doi.org/10.1152/japplphysiol.00145.2023}

  \item[12.] Domingos C., Marôco JL, Miranda M, Silva C, Melo X, Borrego C. Repeatability of Brain Activity as Measured by a 32-Channel EEG System during Resistance Exercise in Healthy Young Adults. \emph{International Journal of Environmental Research and Public Health}. 2023; 20(3):1992. \url{https://doi.org/10.3390/ijerph20031992}

  \item[11.] Brito, J. P., Domingos, C., Pereira, A. F., Moutão, J., \& Oliveira, R. (2022). The Multistage 20-m Shuttle Run Test for Predicting VO2Peak in 6-9-Year-Old Children: A Comparison with VO2Peak Predictive Equations. \emph{Biology}, 11(9), 1356. \url{https://doi.org/10.3390/biology11091356}

  \item[10.] Correia, J. P., Vaz, J. R., Domingos, C., \& Freitas, S. R. (2022). From thinking fast to moving fast: motor control of fast limb movements in healthy individuals. \emph{Reviews in the Neurosciences}. \url{https://doi.org/10.1515/revneuro-2021-0171}

  \item[9.] Rodrigues, F., Domingos, C., Monteiro, D., \& Morouço, P. (2022). A Review on Aging, Sarcopenia, Falls, and Resistance Training in Community-Dwelling Older Adults. \emph{International Journal of Environmental Research and Public Health}, 19(2), 874. \url{https://doi.org/10.3390/ijerph19020874}

  \item[8.] Antunes, A., Domingos, C., Diniz, L., Monteiro, C. P., Espada, M. C., Alves, F. B., \& Reis, J. F. (2022). The Relationship between VO2 and Muscle Deoxygenation Kinetics and Upper Body Repeated Sprint Performance in Trained Judokas and Healthy Individuals. \emph{International Journal of Environmental Research and Public Health}, 19(2), 861. \url{https://doi.org/10.3390/ijerph19020861}

  \item[7.] Santos, J., Ihle, A., Peralta, M., Domingos, C., Gouveia, É. R., Ferrari, G., ... \& Marques, A. (2022). Associations of Physical Activity and Television Viewing With Depressive Symptoms of European Adults. \emph{Frontiers in Public Health}, 19, 20. \url{https://doi.org/10.3389/fpubh.2021.799870}

  \item[6.] Domingos, C., da Silva Caldeira, H., Miranda, M., Melício, F., Rosa, A. C., \& Pereira, J. G. (2021). The Influence of Noise in the Neurofeedback Training Sessions in Student Athletes. \emph{International Journal of Environmental Research and Public Health}, 18(24), 13223. \url{https://doi.org/10.3390/ijerph182413223}

  \item[5.] Domingos, C., Silva, C. M. D., Antunes, A., Prazeres, P., Esteves, I., \& Rosa, A. C. (2021). The influence of an alpha band neurofeedback training in heart rate variability in athletes. \emph{International Journal of Environmental Research and Public Health}, 18(23), 12579. \url{https://doi.org/10.3390/ijerph182312579}

  \item[4.] Domingos, C., Peralta, M., Prazeres, P., Nan, W., Rosa, A., \& Pereira, J. G. (2021). Session Frequency Matters in Neurofeedback Training of Athletes. \emph{Applied Psychophysiology and Biofeedback}, 1-10. \url{https://doi.org/10.1007/s10484-021-09505-3}

  \item[3.] Domingos C, Alves CP, Sousa E, Rosa A and Pereira JG. (2020) Does neurofeedback training improve performance in athletes? \emph{NeuroRegulation}. \url{https://doi.org/10.15540/nr.7.1.8}

  \item[2.] Domingos C, Matias CN, Cyrino E, Sardinha L, Silva A. (2019) Usefulness of Tanita TBF-310 for body composition assessment in Judo elite athletes using a four-compartment molecular model as the reference method. \emph{Journal of the Brazilian Medical Association}. \url{https://doi.org/10.1590/1806-9282.65.10.1283}

  \item[1.] Teixeira FJ, Matias CN, Monteiro CP, Valamatos MJ, Reis JF, Tavares F, Batista A, Domingos C, Alves F, Sardinha LB, Phillips SM. (2019) Leucine Metabolites Do Not Enhance Training-induced Performance or Muscle Thickness. \emph{Medicine \& Science in Sports \& Exercise}, 51(1):56-64 \url{https://doi.org/10.1249/MSS.0000000000001754}
  
\end{enumerate}
